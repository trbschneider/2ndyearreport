%!TEX root = ./report.tex
Spinal Cord Injury (SCI) can have devastating effects on the life of people affected by it. Thanks to innovative treatment strategies, SCI patients can now hope in the concrete possibility of new therapies leading to recovery of feeling and motor functions, with a dramatic repercussion on their future quality of life.

Magnetic Resonance Imaging (MRI) routine scans of the spinal cord are often aiding the diagnosis of SCI, but they have a limited prognostic value because of their qualitative nature and because of their lack of specificity in terms of underlying mechanisms such as inflammation, axonal loss and gliosis. There is the need for in vivo imaging biomarkers for human spinal cord examinations, which are sensitive to tissue changes and which are capable of quantifying underlying structural and functional changes.

The sensitivity of MRI to the diffusion of water molecules in the tissue in vivo has been exploited since the early ’90s for characterising the white matter tissue structure of the brain (3).  The spinal cord is a more challenging structure to study with diffusion imaging because of several problems: the breathing motion, the artefacts arising from the surrounding bones, the pulsation of the cerebro-spinal fluid (CSF) and last but not least its limited size that requires high resolution. Dr. Wheeler-Kingshott  has been at the forefront of spinal cord DT imaging development (8) and tractography applications in patients with Multiple Sclerosis (MS), demonstrating that even at typical clinical field strengths (e.g. 1.5T) it is possible to obtain useful and reliable data, able to discriminate between patient and controls in cross-sectional studies. A figure of the current DT imaging and tractography of the spinal cord is shown in the appendix (page 21). Other groups have also been able to implement protocols for DT imaging of the SC, thanks to technological advances such as multi-channel coils for parallel imaging methods and 3T scanners (9). In consequence, the past couple of years have seen emerging publications suggesting optimised acquisition or new analysis methods for spinal cord diffusion imaging (10).

\section{Problem Statement}
Although some optimisation work for spinal cord applications has already taken place, there is the need for a dedicated effort to develop spinal cord diffusion imaging with the aim of optimising the whole process from the acquisition design to the analysis methods, based on the spinal cord tissue properties and the expected underlying mechanisms of tissue damage and recovery.

\section{Project Aims}
\begin{enumerate}
	\item Improve standard diffusion MRI acquisition and post-processing techniques, focussing on in-vivo imaging characterisation of spinal cord white matter
	\item Develop new imaging and analysis protocols to derive more specific imaging biomarkers that are able to detect partial preservation of long fibre pathways in spared tissue after spinal cord injury and that can distinguish better between axonal damage and functional recovery than existing methods
	\item Evaluate the methods developed in 1. and 2. by performing pilot studies in healthy volunteers and patients with pathological changes in spinal cord white matter
\end{enumerate}
