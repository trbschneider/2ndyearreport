%!TEX root = ./report.tex

\section{Preliminary Investigation of Position Dependency of Radial Diffusivity in the Cervical Spinal Cord}
In this experiment we investigate whether DTI derived parameters are sensitive to the presence of collateral fibres and can be related the axial position of the acquired slice in the spinal cord.
\subsection*{Motivation}
The majority of diffusion MRI studies mainly focused on the longitudinal fibres of the SC and only little is known about the value of DTI for the assessment of the connective collateral fibres. These fibres rise at an angle with the white-matter longitudinal tracts and enter the spinal cord gray matter. They interconnect with other areas of the spinal cord through the central gray matter and form part of many functional connections within the spinal cord (Carpenter, 1991). Recently it has been demonstrated hat the second eigenvector is corresponding to sprouting collateral fibres (Mamata et al., 2006). In this study we focus on DTI of the spinal cord with particular interest in the diffusivity changes caused by the presence of the collateral fibres. We aim to investigate whether these DTI parameters are specific to nerve roots anatomy and therefore have the potential to be used in spinal cord injury to assess the integrity of the axonal connections.
\subsection*{Methods}
\subsubsection{Positioning of DTI scans} Accuracy and reproducibility in slice
positioning for the DW acquisitions is crucial to discriminate differences in diffusion parameters between subjects. Since for this study we wanted to position our slices with respect to the neural foramina, a good visualization of the spinal nerve root is needed. However, finding the exact position of the nerve root is difficult on conventional axial or sagittal MRI scans. We use two sagittal oblique MRI scans to accurately reveal the location of the neuroforamen, similar to (Goodman et al., 2006). Based on a standard axial scan of the cervical spinal cord, we prescribed a sagittal scan that is approximately parallel to the spinal nerve leaving the neuroforamen (see Figure 1A). To visualize the spinal cord and spinal nerve root a second sagittal oblique scan perpendicular to the first one is acquired. This scan is aligned so that at least one slice is parallel to the nerve root (see Figure 1B). The first oblique scan is then used to position axial scans so that one slice intersects the spinal nerve. Figure 2 presents two scans acquired with this positioning. In Figure 2A one can clearly appreciate the neuroforamina between C4 and C7. Furthermore, in Figure 2B the spinal nerve root leaving the spinal cord can be seen. Based on these scans we are able to accurately position the DW scans with respect to the roots anatomy. We assume that the diffusion parameters differ mostly between P1 and P2, i.e. the positions shown in Figure 2C, where P1 coincides with the level of the spinal nerve root leaving the spinal cord and P2 with the vertebral body.

\paragraph{Data acquisition}
Diffusion-weighted scans are acquired on a 1.5T Signa scanner (General Electric Company, Milwaukee, WIS) using a cardiac-gated single shot CO-ZOOM EPI sequence (Dowell et al., 2009) with imaging parameters TR = 5RRs, TE= 95.5ms, voxel size = $1\times 1 \times 5mm^3$and an image matrix of $64\times 64$ (FOV=$13\times 13mm^2$). We acquire 8 distributed diffusion weighted directions (see Table1) interleaved with 4 non-diffusion weighted directions. A b-factor of 1000 $smm^{-2}$ was chosen for optimal DT reconstruction as recommended in(Jones et al., 1999). We focus attention on a single slice acquisition to make sure that the signal from the slice is completely recovered after each shot, given that when using the CO-ZOOM sequence T1 relaxation can affect the signal intensity of subsequent slices in multiple slices acquisition. Also, by positioning one single slice, it is possible to acquire a spinal cord image orthogonally to the main spinal cord fibre direction. To increase signal-to-noise-ratio we initially repeat each scan on each subject 22 times to determine the optimal number of averages needed. Subsequent scans on the same subject are repeated 15 times (see Data analysis). 

\paragraph{Data analysis}
After acquisition, all magnitude images are linearly interpolated to a $128\times128$ matrix on a slice-by-slice basis resulting in an in-plane resolution of 0.5x0.5 mm2. DT reconstruction is performed using the camino toolbox (Cook et al., 2006) and maps of the FA and RD are calculated from the diffusion tensor. In addition we use an alternative method of measuring diffusivity in the axial plane ($ADC_\perp$) from only the 4 co-planar acquisitions with diffusion gradients perpendicular to the spinal cord as described in (Fasano et al., 2009).  The used diffusion directions are marked “*” in Table 1. All calculations are implemented in MATLAB (Mathworks, Natick, MA).

It is well known that in the low SNR regime the diffusion indices are very prone to estimation errors as shown in (Basser and Pierpaoli, 1996; Landman et al., 2008). Thus, for reproducible measurements we need to acquire a sufficient number of averages in each scan. To determine the optimal number of averages for each subject we repeat the diffusion measurements at both slice positions 22 times each (overall scan time was approx. 1h). We then calculate the diffusion indices described above using a subset of the first N repeated measurements with N increasing from 1 to 22. A plot of mean diffusion indices over the spinal cord against the number of averages is presented in Figure 3 for one representative subject. A significant bias can be observed in all diffusion indices when less than 10 averages are used. In none of our subjects significant changes can be seen after 15 repetitions, so we choose the number of averages to be 15 in all subsequent scans.

Due to the relatively long scan time (30-40 minutes for 15 averages) the subject’s position in the scanner is very likely to be affected by motion during the scan. However, registration of spinal cord diffusion data is challenging for several reasons. First of all, diffusion-weighted images typically suffer from low SNR and low tissue contrast especially in the spinal cord. Furthermore, in contrast to the brain, distortion artifacts from surround tissue and breathing motion make it difficult to identify reliable anatomical landmarks in the b=0 images. Moreover, longitudinal symmetry of the spinal cord makes it impossible to correct for motion in this direction. Because of all these confounding factors, we use a restrictive motion model that only corrects for in-plane translation and assumes no movement in the z-direction. We divide data acquired within and between repetitions in different blocks with respect to the interleaved b=0 acquisitions. Each block starts with one b=0 image and contains all DW images up to the next b=0 acquisition. We then co-register two subsequent b=0 images using the VTK CISG registration toolkit (Hartkens et al., 2002). The resulting transformation is then applied to all the images of one block. After registration, we average all scans with corresponding DW from subsequent repetitions and all the b=0 acquisitions individually and perform the diffusion parameter estimation on the averaged data set as described above.

FA, RD and $ADC_\perp$ are then quantified over the whole spinal cord at each position. We semi-automatically segment the cord area on the average b=0 image of each slice using ImageJ  and the YAWI2D  segmentation plug-in. Due to the small size of the cord and the limits in image resolution, the segmented area will inevitably contain contributions from the surrounding CSF. Previous studies have shown that this partial volume effect has considerable effect on estimated diffusion parameters (Alexander et al., 2001). To minimize this effect, we applied a binary erosion filter on the resulting regions of interests. This removes pixels at the edge of the ROI and thus allows removing voxels with CSF contribution from the ROI analysis. Hereby the thickness of the removed edge is dependent on the size of the structure element that is used for the erosion filtering. Figure 3 illustrates the effect of the size of the structure element on the estimated diffusion parameters in one subject. It can be seen that in the initial segmentation, FA is underestimated and diffusivity measurements are overestimated respectively because of the isotropic free diffusion that is present in the voxels with CSF contribution. A structure element of size 2 proves to be sufficient to eliminate the partial volume effect and all measurements reach a stable plateau.

\paragraph{Pilot study}
A pilot study was carried out on 4 healthy female subjects. For each subject, parameter maps of FA, RD and $ADC_\perp$ were calculated as described above. We also calculated colour-coded maps of V1, V2, V3 for each scan. To assess intra-subject scan-rescan reproducibility, the scans were repeated with the same parameters after 5-7 days. Reproducibility of parameters was assessed by computing the coefficient of variation (COV) that is defined as the ratio of the standard deviation $\sigma$ and the mean $\mu$.

\subsection*{Results}
\paragraph{Scan/rescan reproducibility}
Table 2 shows the COV of all measured parameters in all four subjects. It can be seen that our careful approach towards positioning and analyzing the data allows good reproducibility (CoV < 10% in all but one cases) in the scan/rescan experiment among all subjects. Furthermore, it can be noted that parameter variation seems to be slightly elevated in the $ADC_\perp$ parameter compared to RD but differences are negligible.

\paragraph{Single subject position dependency of measured parameters}
Figure 5 compares the measured diffusion parameters between the two investigated positions in all subjects. FA and RD/$ADC_\perp$ are closely dependent, i.e., when FA is low RD and $ADC_\perp$ values are high and increasing FA corresponds to lower RD and $ADC_\perp$ in both positions. This implies that parallel diffusivity in the nerve fibres is position independent and therefore changes of FA between spinal cord levels can be explained by different diffusivities perpendicular to the SC axis. Furthermore, it can be seen that the two methods of measuring diffusivity cross-sectionally give similar values in all subjects apart from minor differences in their standard deviation through the entire section of the cord. This can be explained by the lower number of only 4 diffusion measurements that are used to reconstruct $ADC_\perp$, compared to the 8 diffusion directions used for full DT reconstruction. Since $ADC_\perp$ requires no measurements parallel to the fibre, the number of scans needed for reliable measurements is significantly reduced compared to a DTI acquisition, which can be extremely beneficial for future studies of spinal cord injury patients. 
\paragraph{Between subjects comparision of position dependency of parameters}
Although in individual subjects we can see differences between position 1 and 2 with little variation in parameters between scan and rescan, we find that these trends are not consistent between subjects. In subject 1 and subject 3 we observe lower RD/$ADC_\perp$ and higher FA at nerve root level compared to the vertebral body (see Figure 5). Subject 2 shows an opposite trend with higher FA at spinal root level and lower RD/$ADC_\perp$ respectively. In subject 4 there appear to be no differences between the two positions. It is unclear whether these differences between subjects can simply be explained by normal variation due to physiological noise or if they can be attributed to different fibre architecture in each individual. However, these differences between subjects also become apparent in the direction of the second eigenvector. It has been shown by (Mamata et al., 2006) that the second DT eigenvector is sensitive to the presence of sprouting fibres in the spinal cord. Figure 6 presents the color-coded maps of V2 overlaid on the FA map for two subjects with differing trends in diffusion parameters. Figure 6A displays V2 of subject 2, Figure 6B presents the V2-map of subject 4. In both cases, the first row shows the result from the first scan while the second row shows maps derived from the re-scan. The position of the slice in the second row (i.e. for the rescan) was chosen to correspond anatomically with the position of the slice in the first experiment presented in the first row. This was achieved using our 45° localization scanning method presented above and using a printout of the first scan positioning as reference. It has to be noted that a higher angular in-plane resolution of the diffusion gradient scheme would be needed to allow mapping real anatomical directions of the sprouting peripheral nerves. This however, would increase the number of acquisitions needed and therefore further increase the scan time. Moreover, even with our low-resolution scheme, distinct patterns emerge in each subject in the directions of the second eigenvector and are consistent over the first and second scan. Furthermore, in subject 2,where lower FA and higher RD/$ADC_\perp$ are present at spinal root level compared to mid-vertebra level, we also observe different patterns in position 1 and position 2. In subject 4, which shows no difference in mean diffusion parameters in P1 and P2, the V2 map is also similar in both positions. The same geometry is apparent in the repeated scans for both subjects.

These preliminary findings suggest that the diffusion measurements in the spinal cord depend indeed on the presence of sprouting fibres. However, the organization of those fibres seems to be varying between subjects and needs to be addressed. The consistency of the patterns at different slice positioning between scans within each subject is encouraging because it suggests that DT parameters, and in particular RD/$ADC_\perp$, can be used in longitudinal studies to assess structural changes due to degeneration or regeneration of fibers. Table 2 shows the COV of all measured parameters in all four subjects. It can be seen that our careful approach towards positioning and analyzing the data allows good reproducibility (CoV < 10$\%$ in all but one cases) in the scan/rescan experiment among all subjects. Furthermore, it can be noted that parameter variation seems to be slightly elevated in the $ADC_\perp$ parameter compared to RD but differences are negligible.

\subsection*{Conclusion}
This study investigated the position dependency of diffusion parameters measured in the cervical spinal cord at two distinct levels. We concentrated on optimizing the acquisition protocol and the analysis procedure to eliminate the various confounding effects, e.g. from uncertainty in slice positioning, presence of physiological noise and subject motion as well as parameter estimation errors from partial volume effect. Furthermore, we were able to find differences between the two investigated positions, consistently reproduced within subjects. Studies using RD/$ADC_\perp$ measurements in the spinal cord will have to take into consideration the inter-subject variability of these parameters.
 
\section{Fuzzy partial volume correction of spinal cord DTI parameters}
In this section we present a novel correction method for partial volume effects on the estimation of DTI parameters in the cervical spinal cord.
\subsection*{Introduction}
Due to the small size of the cord and the limited spatial resolution, a large proportion of voxels are affected by partial volume averaging (PVA) from surrounding cerebro-spinal fluid (CSF). Water molecules in CSF are less hindered than in nervous tissue, resulting in increased diffusivity measures and decreased anisotropy in PVA voxels [3,4]. This can lead to biased average measurements over specific regions of interest (ROIs) and over the whole cord volume and potentially conceal subtle disease effects. Existing correction methods like [5] are often not applicable due to the low signal-to-noise ratio in spinal cord diffusion images. Therefore in common practice, CSF affected voxels are excluded from analysis with a subjective and manual editing of the outlined ROIs. However, objectively deciding which voxels to exclude while retaining information can be problematic, particularly when the cord area is small and only few unaffected voxels exist, e.g., in patients with spinal cord atrophy. We introduce a robust partial volume correction method for average DTI parameters that avoids the manual exclusion of PVA affected voxels. Instead, we introduce a contribution weighting factor for each affected voxel that depends of on its distance to the interface between spinal cord voxels and CSF. We investigate the accuracy of our approach in healthy volunteers and demonstrate that our method significantly reduces PVA effects on mean DTI indices.
\subsection*{Methods}
\paragraph{Data acquisition and DTI analysis}
We acquired diffusion-weighted images of 14 healthy volunteers (13 male, age=35±11). In each subject cardiac gated DTI of the cervical cord was performed (acquisition matrix=96x96, sinc interpolated in image space to $192\times 192$, FOV=$144\times 144mm^2$, slice thickness=5mm, 20 slices, TE=88ms, TR≈4000ms) with a total of 100 b=1000s/mm$^2$ diffusion weighted volumes (20 unique diffusion directions repeated 5 times) and 5 non-diffusion weighted volumes. In each voxel the diffusion tensor was fitted to the data using camino  (www.camino.org.uk) [6] and maps of fractional anisotropy (FA) and mean diffusivity (MD), axial diffusivity (AD) and radial diffusivity (RD) were generated. 
\paragraph{PVA method}
We semi-automatically delineate the cervical cord between levels C1/2 and C4/5 using the active surface segmentation [7] available in Jim6 (www.xinapse.com), performed on the computed FA maps [8]. A 2D distance transformation is applied to the binary segmentation masks, i.e., determining the distance d of each masked voxel to the border of the mask. Assuming that only voxels close to CSF are affected by PVA, the fuzzy partial volume correction factor $w$ is then computed as: 

$$
x =\left\{\begin{array}{lll}
							d/\mbox{max}(d)&\mbox{ if } d\leq c\\
							1&\mbox{ otherwise }
		  \end{array}
   \right.
$$
where $c$ is a cutoff distance determined on the basis of the DTI parameter values (see Figure 1). This approach ensures that for larger spinal cord areas, the border voxels are weighted less than in the case of small cord areas. The weighted average using the weighting factors w is computed for all DTI parameters over the whole segmented spinal cord area. We determine the optimal cutoff voxel distance $c'$ in our dataset so that for $c\geq c'$ the average DTI parameters over the cord area reach a stable plateau, i.e., assuming that CSF contribution effects are minimized. In our data, DTI parameters reach the desired plateau for $c\geq 2$ voxels (see Figure 2) and are in agreement with previously reported values in the healthy cord [8,9]. Thus the cutoff value $c=2$ is chosen for further analysis. A two-tailed paired t-test is performed to compare significance of differences between uncorrected and corrected measurements among all subjects.
\subsection*{Results and discussion}
Table 1 shows lower standard deviation of diffusivity parameters among subjects when using our PVA correction method, suggesting lower inter-subject variability compared to the uncorrected measurements. Furthermore, the largest reduction of DTI values is observed in the RD (p<0.0001). We also find moderate decrease in the AD and MD and increase in FA (all p<0.0001). These results can be explained by CSF contribution to average measurements in uncorrected values and are in agreement with similar findings in simulations [3] and in the brain [4]. 
\subsection*{Conclusion}
In this study we propose a novel fuzzy partial volume correction method that removes CSF contribution effects in measurements of DTI parameters over the whole spinal cord volume. We avoid fully excluding all potentially CSF contaminated voxels, and introduce a weighting factor that is dependent on the size of the cord and therefore accounts for the variability in number of white matter voxels. This allows more reliable measurements, particularly in patients who might suffer from white matter atrophy. Our method can be easily extended to other analysis methods such as histogram analysis and other quantitative modalities such as magnetization transfer imaging.

\section{Fuzzy partial volume correction of spinal cord DTI parameters}
\subsection*{Introduction}
In this study we investigate accuracy and sensitivity of tract-specific q-space imaging (QSI) metrics in healthy controls. The principle of QSI is to exploit the inverse Fourier relation between the signal S(q) and the displacement density function (DPDF) p(r), with q being the diffusion wave number and r being the average displacement of water molecules [1]. Studies on experimental MRI systems have shown that QSI provides accurate information about microscopic restriction in nervous tissue [2,3]. Although the conditions for true QSI, such as the short gradient pulse, are impossible to achieve in clinical systems due to hardware limitations, QSI metrics still provide novel imaging contrasts, as demonstrated in [4,5]. The clinical potential of q-space metrics in the assessment of white matter diseases has been shown in the brain [6] and spinal cord [7]. However, most clinical QSI studies only focused on a small number of patients and failed to demonstrate the reliability of QSI. The aim of this study is to report reproducibility of QSI metrics in the cervical spinal cord on a standard 3T clinical MRI scanner. We also assessed QSI measures both in-plane (XY) and parallel to the main spinal cord axis (Z), not presented before. We compare QSI measures derived in gray matter and different ascending and descending white tracts of the cervical spinal cord in healthy subjects and investigate associations between QSI parameters and conventional apparent diffusion coefficient (ADC) measures, both in plane and along the cord.                                          
\subsection*{Methods}
\paragraph{Study design \& Data acquisition}: We recruited 9 right-handed male healthy subjects (mean age 35±11yrs) to be scanned on a 3T Tim Trio (Siemens Healthcare, Erlangen). Three subjects were recalled for a second scan on a different day to assess intra-subject reproducibility of QSI derived parameters. We performed cardiac-gated high b-value axial DWI (matrix=96x96, b-spline interpolated to 192x192 in image space, FOV=144x144mm2, slice thickness=5mm, 20 slices, TE=110ms, TR≈4000ms) with 32 b values between 0-3000s/mm2 in b=50s/mm2 steps (gradient duration=45ms, diffusion time=55ms, maximum gradient strength=23mT/m). Three different DWI directions were acquired: two directions perpendicular (XY) and one parallel (Z) to the main spinal cord axis. The two perpendicular diffusion directions were averaged to increase the signal-to-noise ratio. The measurements were linearly regridded to be equidistant in q-space and the DPDF was computed using inverse fast Fourier transformation. To increase the resolution of the DPDF, the signal was extrapolated in q-space to a maximum q=166mm-1 by fitting a bi-exponential decay curve to the DWI data as in [1,7]. Maps of the full width at half maximum (FWHM) and zero displacement probability (P0) were derived for XY and Z. For comparison we also computed the apparent diffusion coefficient (ADC) from the monoexponential part of the decay curve (b≤1100s/mm2) as in [7] for both XY and Z directions.    
ROI analysis: We semi-automatically delineate the whole cervical spinal cord area (SCA) between levels C1 and C3 on the b=0 images using the active surface segmentation [8] available in Jim6 (www.xinapse.com). On the segmentation mask we perform a morphological erosion (2 iterations) to exclude voxels with potential partial-volume average effect from surrounding cerebro-spinal fluid (CSF). In addition, four regions of interest (ROI) were manually placed in specific white matter tracts and one ROI was positioned in the gray matter on all slices between level C1 and C3. The four white matter regions comprised the left and right tracts (l\&r-LT) running in the lateral columns and the anterior (AT) and posterior tracts (PT) similar to [9]. 
Statistical processing: We report reproducibility as the intra-subject coefficient of variation (COV=SD measurements/mean measurements) for the three scan/rescan subjects and the inter-subject COV among all nine subjects. Further, we compare significant differences in the group mean values of the ADC parameters (ADCxy, ADCz) and QSI metrics (P0xy, P0z, FWHMxy, FWHMz) between tracts by performing the Hotellings-T2 test (confidence interval=99\%). To investigate the relevance of measurements in the Z direction, we compute the same significance test of XY-only QSI parameters (P0xy, FWHMxy). Finally, we investigate the relationship between individual ADC and QSI measurements in XY and Z directions for each tract using the Spearman’s rho correlation coefficient. 
\subsection*{Results}
Reproducibility: In both intra-subject scan/rescan experiments and among subjects we observe a consistently lower COV in QSI metrics compared to ADC measurements (see Figure 1). In particular, ADCxy shows the largest intra- and inter-subject variation (>25\%) in most tracts. In contrast, tract-specific QSI measurements vary less, and the majority of observed CoVs are between 5-10%. 
Tract-specific differences: Figure 2 reports QSI and ADC values among all 9 subjects. We find significant group differences in QSI and ADC parameters between different white matter tracts. Figure 3 illustrates the tract-specific differences DPDFs in one exemplary subject. In particular, there are significant differences in the ADCxy and ADCz between the PT, AT and lateral tracts (p<0.01) that are not observable with QSI parameters. On the other hand, XY and Z QSI parameters showed significant differences between left and right lateral tracts (p<0.01), as well as differences between AT and l-LT (p<0.05). However, perpendicular QSI metrics alone do not show significant differences in any white matter tract. Both ADC and QSI metrics are significantly different between white matter tracts and gray matter (p<0.001).
QSI – ADC correlation: We further observe significant correlations between ADC and QSI parameters in both XY and Z direction in all tracts.  In XY direction, the strongest associations between FWHMxy and ADCx are found in AT and PT (p<0.001, rho≈0.8), although weaker correlations are also found in the r-LT (p<0.05, rho=0.66). In Z direction, we find strong positive correlations only in PT and l-LT between ADCz and FWHMz (p<0.001, rho≈0.9) and negative correlation with P0z (p<0.001,rho≈-0.8) respectively. Over the whole SCA, we found correlation between all XY and Z measurements: the strongest correlation is found between ADCz and P0z (p<0.01, rho=-0.9) and a weak correlation is found between ADCxy and FWHMxy(p<0.05,rho≈0.7) and P0xy(p<0.05, rho≈-0.7).
\subsection*{Discussion \& Conclusion}
QSI metrics obtained without sequence development, using standard DWI protocol available on a 3T clinical scanner, show a good reproducibility that is superior to simple ADC analysis.  We observe tract-specific correlations between ADC and QSI parameters. However, especially in the lateral tracts, associations are weaker than in the anterior and posterior tracts, suggesting additional information in both XY and Z from QSI analysis in these columns. We further demonstrate that QSI parameters provides complementary metrics that allow discrimination of white matter tracts in healthy controls that cannot be distinguished with ADC alone. Our findings also suggest that the Z direction provides additional information to perpendicular measurements.


